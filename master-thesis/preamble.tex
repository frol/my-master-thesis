% Copyright 2011 Zoresvit (c) <zoresvit@gmail.com>
% Copyright 2012-2013 Frol (c) <frolvlad@gmail.com>
% 
% Latex preambule for regular reports and scientific documents. 
% The format is close to one described in DSTU-3008-95 but does not completely
% satisfy all its requirements.
% Listings package enabled and set up for source code pasting.
%/

\documentclass[a4paper,14pt]{styles/dstu3008_95}
%\documentclass[utf8,usehyperref,12pt]{G7-32}
%\usepackage[T2A]{fontenc}
%\usepackage[utf8]{inputenc} %% ваша любимая кодировка здесь
%\usepackage[english,russian]{babel}
%\usepackage{float}
%\usepackage[dvips]{graphicx}
%\graphicspath{{graphics/}}
%\usepackage{listingsutf8} 	% insert source code listings

\usepackage[utf8]{inputenc}
\usepackage[T2A]{fontenc}
%\usepackage{literat}
\usepackage[english, russian]{babel}
\usepackage[unicode,linktocpage=true]{hyperref}

%\makeatletter
%\renewcommand{\@makechapterhead}[1]{%
%\vspace*{50 pt}%
%{\setlength{\parindent}{0pt} \raggedright \normalfont
%\bfseries\Huge\thechapter.\ #1
%\par\nobreak\vspace{40 pt}}}
%\makeatother

\usepackage[russian]{nomencl}
\makenomenclature
\renewcommand{\nomname}{Список сокращений, условных обозначений, символов, единиц и терминов}
\newcommand{\abbr}[2]{\nomenclature{#1}{ --- #2}}

\usepackage{amsmath}
\usepackage{amsfonts}
\usepackage{amssymb}
\usepackage{amstext}
\usepackage{mathtools}
\DeclarePairedDelimiter{\ceil}{\lceil}{\rceil}

\usepackage{enumerate}      % for changing enumeration appearance
\usepackage{graphicx}		% including pictures and graphics to the document
%\usepackage[pdftex]{graphicx}
\usepackage{indentfirst}	% for indenting first paragraph line
\usepackage{wrapfig}		% for text wrapping around pictures
\usepackage{pscyr}			% Cyrillic fonts
\usepackage{gensymb}		% degree and Celsius sign using
%\usepackage{cite}			% citation from bibliography
\usepackage[sort&compress,numbers,square,colon]{natbib}         % citation from bibliography
%\usepackage{caption}		% caption customizations
\usepackage{caption}[2011/08/06]
\usepackage{setspace}		% set line spacing
\usepackage{fancyhdr}		% for customizing section headings format
%\usepackage{tocloft}		% customize table of contents settings
\usepackage{geometry}		% page customizations
%\usepackage{sectsty}		% customize sections heading style
\usepackage{listingsutf8} 	% insert source code listings
\usepackage{styles/algorithm2e}    % insert algorithm code
\usepackage{appendix}       % enable appendices at the end of document
\usepackage{rotating}		% for sideways figures and tables
\usepackage{longtable}		% for wrapping tables over multiple pages
\usepackage{array}          % multiple row in math formulas
\usepackage[usenames,dvipsnames]{color}  % custom color defining
\usepackage{multirow}       % merge multirow in tables
%\usepackage[none]{hyphenat} % disable word breaking
\usepackage{hyphenat} % word breaking
\sloppy                     % deny words cross right margin
\usepackage{colortbl}       % highlight cells, rows, columns in tables
\usepackage{enumitem}       % setitemsize feature for ordered/unordered lists
\usepackage{titlesec}

% ===[ Page settings ]===

\geometry{left = 2.5cm, right = 1.5cm, top = 2cm, bottom = 2cm}

% ===[ Text format settings ]===

\onehalfspacing     % set 1.5 line spacing interval
%\linespread{1.17}   % There is strange definition of line spacing - "exact 18"
\setlength{\parindent}{1.25cm}  % set paragraph first line indentation
\setitemize[1]{leftmargin=0cm,itemindent=2cm,listparindent=-2cm}
\setitemize[2]{leftmargin=1.25cm,itemindent=2cm,listparindent=-2cm}
\setenumerate[1]{leftmargin=0cm,itemindent=2cm,listparindent=-2cm}
\titleformat{\section}[block]{\hspace{\parindent}}{\thesection}{0.5em}{}
\titlespacing{\section}{0pt}{1em}{1em}
\titleformat{\subsection}[block]{\hspace{\parindent}}{\thesubsection}{0.5em}{}
\titlespacing{\subsection}{0pt}{1em}{0em}

\renewcommand{\thefootnote}{\arabic{footnote})}  % Add brackets to footnote numberings

%\addto\captionsukrainian{\def\contentsname{\centerline{ЗМІСТ}}}
%\addto\captionsukrainian{\def\refname{\centerline{ПЕРЕЛІК ПОСИЛАНЬ}}}

%\renewcommand{\appendixpagename}{\begin{center} Додатки \end{center}}
%\renewcommand{\appendixtocname}{Додатки}

% ===[ customizing section headings ]===

%\sectionfont{\centering \large}
%\subsectionfont{\normalsize}

% ===[ Caption Settings ]===

\DeclareCaptionLabelSeparator{hyphen}{ -- }
\captionsetup[figure]{labelsep=hyphen, position=bottom, justification=centering, figurename=Рисунок}
\captionsetup[table]{format=plain, belowskip=4pt, labelsep=hyphen, position=top, justification=justified, singlelinecheck=off, tablename=\mbox{\hspace{1.25cm}Таблица}}

% ===[ Bibliography Settings ]===

\bibliographystyle{unsrt}

% ===[ Switch enumerations to digit.digit format ]===

\renewcommand{\theenumi}{\arabic{enumi})}
\renewcommand{\labelenumi}{\arabic{enumi})}
\renewcommand{\theenumii}{\arabic{enumii})}
\renewcommand{\labelenumii}{\arabic{enumi}.\arabic{enumii}}
\renewcommand{\theenumiii}{\arabic{enumiii}}
\renewcommand{\labelenumiii}{\arabic{enumi}.\arabic{enumii}.\arabic{enumiii}}

% ===[ Listings package settings ]===
\DeclareCaptionFont{white}{\color{black}}
\DeclareCaptionFormat{listing}{\colorbox[rgb]{0.85, 0.85, 0.99}{\parbox{\textwidth}{\hspace{10pt}#1#2#3}}}
\def\lstlistingname{Листинг}
\renewcommand*\arraystretch{1.5} % table vertical padding

\captionsetup[lstlisting] {
format=listing,
labelfont=white,
textfont=white,
singlelinecheck=true,
margin=0pt,
font={bf,footnotesize}
}

\lstset {
extendedchars=false,
inputencoding=utf8, 
keepspaces=true, 
language=bash, 
basicstyle=\footnotesize,       % the size of the fonts that are used for the code
keywordstyle=\textbf,
xleftmargin=13pt,
framexleftmargin=10pt,
framexrightmargin=3pt,
framexbottommargin=4pt,
frame=single,	          % adds a frame around the code
captionpos=t, 			  % place position at top
belowcaptionskip=7pt,
tabsize=4,	              % sets default tabsize to 4 spaces
showspaces=false,         % show spaces adding particular underscores
showstringspaces=false,   % underline spaces within strings
showtabs=false,           % show tabs within strings adding particular underscores
breaklines=true,          % sets automatic line breaking
numbers=left,             % where to put the line-numbers
numberstyle=\scriptsize,  % the size of the fonts that are used for the line-numbers
stepnumber=1,  % the step between two line-numbers. If it's 1 each line will be numbered
numbersep=15pt            % how far the line-numbers are from the code
}

% ===[ Algorithm2e package settings ]===
\SetKwSty{textmd}         % Disable bolding in algorithm

% ===[ extra aliases ]===
\newcommand*\xor{\oplus}
\newcommand*\xorsum{\bigoplus}
\newcolumntype{C}[1]{>{\centering\let\newline\\\arraybackslash\hspace{0pt}}m{#1}}
\newcolumntype{L}[1]{>{\raggedright\let\newline\\\arraybackslash\hspace{0pt}}m{#1}}
\newcolumntype{R}[1]{>{\raggedleft\let\newline\\\arraybackslash\hspace{0pt}}m{#1}}

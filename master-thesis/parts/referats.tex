\Referat{Реферат}

% TODO поправить числа
Магистерская аттестационная работа содержит 75 страниц, 7 рисунков, 16 таблиц,
57 формул, 28 источников, 4 приложения.

В данной работе рассмотрены методы формирования нелинейных узлов замен, широко
используемых в блочных симметричных шифрах. Актуальной задачей является
совершенствование симметричных средств защиты информации путем разработки новых
подходов и способов построения нелинейных узлов замен с улучшенными свойствами.

Объект исследования --- процесс построения нелинейных узлов замен блочных
симметричных шифров.

Предмет исследования --- метод имитации отжига для построения нелинейных узлов
замен блочных симметричных шифров.

Цель --- повышение криптографических свойств формируемых нелинейных узлов замен
на основе совершенствования метода имитации отжига.

\bigskip\bigskip

\begin{hyphenrules}{nohyphenation}
НЕЛИНЕЙНЫЙ УЗЕЛ ЗАМЕН, НЕЛИНЕЙНОСТЬ, АВТОКОРРЕЛЯЦИЯ, S-BOX, ВЕСОВЫЕ
КОЭФФИЦИЕНТЫ, МЕТОД ИМИТАЦИИ ОТЖИГА
\end{hyphenrules}

\Referat{Abstract}

Master's thesis includes 75 pages, 7 pictures, 16 tables, 57 formulas,
28 sources, 4 appendixes.

In this work considered the methods for S-box construction, that are widely used
in block symmetric ciphers. The urgent task is to improve the symmetric
information security tools through the development new approaches and methods
constructing S-boxes with improved properties.

Research object is the process of building S-boxes for block symmetric ciphers.

Research subject is the method of simulating annealing for S-box building for
block symmetric ciphers.

The object is the enchancement of cryptographic properties of built S-boxes that
are based on elaboration the simulation annealing method.

\bigskip\bigskip

\begin{hyphenrules}{nohyphenation}
S-BOX, NONLINEARITY, AUTOCORRELATION, WEIGHT COEFFICIENTS, SIMULATED ANNEALING
METHOD
\end{hyphenrules}

\Referat{Реферат}

Магістерська атестаційна робота містить 75 сторінок, 7 рисунків, 16 таблиць,
57 формул, 28 джерел, 4 додатки.

В даній роботі розгянуто методи формування нелінійних вузлів замін, які широко
використовуются у блочних симетричних шифрах. Актуальнім завданням є
вдосконалення симетричних засобів інформаційної безпеки шляхом розробки нових
методів та способів формування нелінійних вузлів замін з покращенними
властивостями.

Об’єкт дослідження --- процес побудови нелінійних вузлів замін блочних
симетричних шифрів.

Предмет дослідження --- метод імітації відпалу для побудови нелінійних вузлів
замін блочних симетричних шифрів.

Мета --- підвищення криптографічних властивостей нелінійних вузлів замін
заснованих на вдосконаленні методу імітації відпалу.

\bigskip\bigskip

\begin{hyphenrules}{nohyphenation}
НЕЛІНІЙНІ ВУЗЛИ ЗАМІН, НЕЛІНІЙНІСТЬ, АВТОКОРЕЛЯЦІЯ, S-BOX, ВАГОВІ КОЕФІЦІЄНТИ,
МЕТОД ІМІТУВАННЯ ВІДПАЛУ
\end{hyphenrules}

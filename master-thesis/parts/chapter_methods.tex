\chapter{Анализ методов синтеза регулярных нелинейных узлов замен}

Анализ открытой литературы показал, что на сегодняшний день существует ряд
вычислительных методов синтеза регулярных нелинейных узлов замен, которые можно
разделить на три основных класса \cite{Burnett,Connor,Millian1,Millian2,Clark1,Laskari,Tesar}:

\begin{enumerate}
    \item методы случайного поиска (побитовые методы (bit-by-bit methods),
        методы случайной генерации с фильтрацией (random generation))
    \item методы алгебраического построения (степенное отображение в поле,
        инверсия в поле с афинным преобразованием)
    \item методы эвристического поиска (метод градиентного подъема (hill climbing),
        генетические алгоритмы (genetic algorithms),
        метод имитации отжига (simulated annealing),
        метод дифференциальной эволюции (differential evolution),
        метод оптимизации роем частиц (particle swarm optimization))
\end{enumerate}

В данной работе рассматривается метод эвристического поиска, метод имитации
отжига, из соображений достаточно хороших характеристик синтезируемых S-блоков
за небольшое время. Так, например, на основе проведенных в работах
\cite{Clark1,Kavut} сравнений показано, что использование метода имитации отжига
позволяет реализовать вычислительный поиск криптографических функций с лучшими
на сегодняшний день показателями. Описание и алгоритм метода приведены в
разделе~\ref{section:simulated_annealing}.

\section{Методы случайной генерации}

Метод случайной генерации с фильтрацией.

Производится формирование нелинейного узла замен случайным образом, а затем
оценивается криптографические показатели, которыми обладает сформированный
S-блок. Если сформированный узел замен не удовлетворяет накладываемым критериям
поиска (отбора), формируется следующий узел замен. Данный метод является крайне
неэффективным.

Побитовый метод.

На вход алгоритма подается система ограничений на криптографические показатели
отбираемых $m$ булевых функций (и соответственно всех их линейных комбинаций).

Некоторая функция $j$, $j \leq m$, фиксируется и считается отобранной для
нелинейного узла замен, если функция и все ее линейные комбинации с уже
отобранными функциями удовлетворяет заданной системе криптографических
ограничений. Если функция не удовлетворяет наложенным криптографическим
критериям отбора, она отбрасывается и формируется следующая. Алгоритм продолжает
свое выполнение до тех пор, пока требуемый S-блок $n \times m$ не будет сформирован.

Выходными данными алгоритма являются $m$ булевых функций с требуемыми
криптографическими показателями, непосредственно представляющие собой нелинейный
узел замен.

\section{Метод имитации отжига}
\label{section:simulated_annealing}

Метод имитации отжига заключается в вероятностном вычислительном поиске
криптографических нелинейных узлов замен.

Поиск начинается с некоторого начального состояния $S=S_0$. Параметр $T$ "---
некий контрольный параметр, известный как температура. $T$ инициализируется
высокой температурой $T_0$ и постепенно снижается. При каждом значении
температуры, выполняется определенное число $MIL$ (Moves in Inner Loop, шагов во
внутреннем цикле) шагов к новым состояниям. Состояние-кандидата $Y$ выбирается
случайным образом из соседей $N(S)$ текущего состояния. Вычисляется изменение
значения функции cost, $\delta=cost|Y| - cost(S)$. Если значение $cost(S)$
улучшается (т.е. $\delta < 0$ для задачи минимизации), тогда выполняется шаг
относительно этого состояния ($S = Y$); в противном случае "--- он выполняется с
некоторой вероятностью. Чем хуже шаг, тем меньше вероятность того, что он будет
принят; чем ниже температура $T$, тем менее вероятно, что ухудшающий шаг будет
принят. Вероятностное принятие решения определяется генерацией случайного числа
$U$ в интервале $(0..1)$ и выполнением указанного ниже сравнения.

Вначале температура высокая и принимается почти каждый шаг. Это сделано для
того, чтобы поиск носил не локальный, а глобальный характер. По мере того как
температура уменьшается, становится все более трудно принимать ухудшающие шаги.
В конце концов, допускаются только улучшающие шаги и процесс застывает. Алгоритм
прерывается, когда встречается критерий остановки. Общий критерий остановки "---
остановка поиска при достижении заданного числа $MaxIL$ внутренних циклов, либо
когда было выполнено некоторое максимальное число $MUL$ последовательных
непродуктивных внутренних циклов (т.е. без единого принятого шага). При этом
лучшее достигнутое состояние сохраняется, поскольку поиск может выйти из него и
впоследствии не найти состояние с подобными показателями. В конце каждого
внутреннего цикла температура понижается. В исследуемом алгоритме использовалось
геометрическое охлаждение "--- умножение на константу охлаждения $\alpha$ в
интервале $(0..1)$.

Алгоритм имитации отжига SA:

\begin{algorithm}
    \SetAlgoNoLine
    \RestyleAlgo{plain}
    $S \leftarrow S0$\;
    $T \leftarrow T0$\;
    \Repeat{критерий остановки не достигнут}{
        \For{$i \leftarrow 1$ \KwTo $MIL$}{
            выбрать $Y \in N(S)$\;
            $delta \leftarrow cost(Y) - cost(S)$\;
            \eIf{delta < 0}{
                $S \leftarrow Y$\;
            }{
                сгенерировать $U \leftarrow U(0,1)$\;
            }
            \If{U < exp(-delta/T)}{
                $S \leftarrow Y$\;
            }
        }
        $T \leftarrow T * alpha$\;
    }
\end{algorithm}

Соседей функции $f$ можно определить следующим образом. Функция $g$ находится по
соседству с функцией $f$, если:
\begin{equation}\begin{split}
\exists x,y \in Z^{n}_{2}&: \hat{f}(x) \neq \hat{f}(y), \hat{g}(x) = \hat{f}(y), \hat{g}(y) = \hat{f}(x), \\ 
& \forall z \in Z^{n}_{2} \backslash {x,y}: \hat{g}(z) = \hat{f}(x).
\end{split}\end{equation}

Поиск начинался со сбалансированной, но при этом случайной функции. Один шаг
алгоритма меняет местами два отличных элемента таблицы истинности функции,
сохраняя ее сбалансированность.

\section{Стойкость блочных симметричных шифров относительно дифференциального криптоанализа}

Метод дифференциального криптоанализа был основан Бихамом и Шамиром и
применяется как атака с использованием выбранного открытого текста на блочные
симметричные шифры. Дифференциальная атака на БСШ включает анализ переходов
входных разностей открытого текста в соответствующие выходные разности
шифртекста. Эти переходы используются с целью получения информации "--- биты
ключа, что может снизить вычислительную сложность взлома шифра.

Пусть для блочного шифра с длиной блока $B$ бит блоки $X = X_1 X_2 X_3 \ldots
X_B$ и $Y = Y_1 Y_2 Y_3 \ldots Y_B$ представляют собой блоки открытого текста и
шифртекста соответственно. Если $X_i$ и $X_j$ – два $B$-битных блока открытого
текста, $Y_i$ и $Y_j$ "--- их соответствующие выходные блоки, то $\Delta X = X_i
\xor X_j$ и $\Delta Y = Y_i \xor Y_j$ "--- их входные и выходные разности
соответственно. Пара входных и выходных разностей называется дифференциалом.
Каждому дифференциалу соответствует вероятность, что выходная разность появится
определенное число раз при заданной входной разности, т.е. $P(\Delta Y | \Delta
X)$. Чем ниже дифференциальная вероятность, тем менее вероятно, что выходная
разность появится для определенной входной разности. Это желательно, поскольку
мы стремимся минимизировать корреляцию между входными и выходными разностями с
целью усложнения точного предсказания промежуточных бит на протяжении процесса
шифрования. Серия дифференциалов для последовательности раундов в шифре, которые
удовлетворяют $\Delta^k_Y = \Delta^{k+1}_X$ для раундов от 1 до $l$, называется
$l$-раундовой дифференциальной характеристикой. Дифференциальные характеристики
используются для определения общей дифференциальной вероятности шифра. Таким
образом, вероятность дифференциальной характеристики может быть измерена
вычислением произведения вероятностей дифференциала каждого отдельного раунда,
предполагая при этом, что они независимы друг от друга.

Очевидно, что на значения дифференциальных вероятностей влияют используемые
компоненты шифра. Нелинейные узлы замен являются ключевым компонентом блочных
шифров, поскольку являются основным и, зачастую, единственным источником
нелинейности системы шифрования. Дифференциал S-блока "--- два значения,
представляющие собой разность между двумя входными значениями и разность между
их соответствующими выходными значениями. Для S-блока $n \times m$ существует
всего $2^{2n-1} - 2^{n-1}$ возможных отдельных входных разностных пар. Таблица
частот появления всех результирующих выходных разностей (для каждого значения
входной разности возможны $2^m$ различных значений) является таблицей
распределения дифференциалов S-блока. Таким образом, дифференциальная таблица
"--- это матрица $2^n \times 2^m$, содержащая частоту появления всех возможных
выходных разностей при каждой возможной заданной входной разности. Наибольшее
значение в дифференциальной таблице S-блока называют $\Delta$-равномерностью.

Сумма значений каждой строки в дифференциальной таблице должна равняться $2^n$.
Поэтому плоская дифференциальная таблица, т.е. таблица, в которой частота
значений равномерно распределена, означает, что величина частот невелика. Узел
замен, чья дифференциальная таблица является плоской, практически не дает
никакой информации о выходных разностях, которая может быть применена для
раскрытия промежуточных бит шифра. Большие же частотные значения в таблице
дифференциалов могут быть использованы для формирования дифференциальной
характеристики с высокой вероятностью.

В традиционном блочном шифре в раундах преобразования S-блоки осуществляют
множество замен. Объединяя дифференциалы S-блоков, может быть получена
дифференциальная характеристика для всего шифра. Для того чтобы шифр был стоек
относительно дифференциального криптоанализа, вероятность его дифференциальной
характеристики должна быть маленькой. Шифры, содержащие большее число раундов,
более вероятно достигнут низкой вероятности дифференциальной характеристики.
Величина дифференциалов S-блока также влияет на вероятность дифференциальной
характеристики всего шифра. Отсутствие каких-либо больших значений в таблице
распределения разностей S-блока приводит к маленьким дифференциальным
вероятностям S-блока и, таким образом, производит дифференциальную
характеристику с малой вероятностью.

Пусть $DDT_S$ (Difference Distribution Table) "--- матрица $2^n \times 2^m$,
представляющая таблицу распределения разностей S-блока $S$ $n \times m$. Пусть
$A_S$ "--- матрица $2^n \times 2^m$, представляющая автокорреляционную матрицу
S-блока $S$. Нижняя граница дифференциальной $\Delta$-равномерности, выражаемой
через максимальное абсолютное значение автокорреляционной матрицы S-блока,
задается следующим образом:
\begin{equation}\Delta \geq 2^{n - m} + 2^{-m} AC(S_{n,m}),\end{equation}
где $AC(S_{n,m})$ "--- автокорреляция S-блока $S$.

Дальнейшее наблюдение, сделанное в~\cite{Burnett}, заключается в том, что
маленькое значение $\Delta$-равномерности подразумевает маленькое значение
$AC(S_{n,m})$. Следовательно, минимизация автокорреляции S-блоков способствует
повышению стойкости шифра относительно дифференциального криптоанализа через
минимизацию их дифференциальной равномерности и, в свою очередь, сокращение
вероятности дифференциальной характеристики шифра.

В~\cite{Burnett} представлены две верхние границы для нелинейности S-блока $n
\times m$, которые относятся к подсчету ненулевых ячеек в дифференциальной
таблице S-блока и зависят от $n$ и $m$. По существу, увеличение числа ненулевых
ячеек в таблице соответствует S-блоку с потенциально высокой нелинейностью.
Таким образом, использование высоко нелинейного S-блока повышает минимальное
число ненулевых ячеек в дифференциальной таблице, сокращая восприимчивость шифра
к дифференциальному криптоанализу. В 1990-х гг. было показано~\cite{Biham}, что
шифр DES (Data Encryption Standard) может быть взломан дифференциальным
криптоанализом. Восприимчивость DES была обусловлена, в первую очередь, тем, что
дифференциальные таблицы S-блоков DES обладают явной неравномерностью, в то
время как стойкость относительно дифференциального криптоанализа характеризуется
равномерностью дифференциальной таблицы.

Подводя итоги, можно сказать, что стойкость шифрующих систем относительно
дифференциального криптоанализа усиливается с использованием в них S-блоков с
показателями высокой нелинейности и низкой автокорреляции.

\section{Стойкость блочных симметричных шифров относительно линейного криптоанализа}

Линейный криптоанализ, предложенный Мацуи [96] в 1993 г., является атакой по
известному открытому тексту, которая стремится аппроксимировать отношения между
битами открытого текста, шифр-текста и ключа через построение линейного
выражения и оценку вероятности этого выражения, точно отображающего это
отношение. Таким образом, целью линейного криптоанализа является раскрытие битов
ключа.

Пусть для блочного шифра с длиной блока $B$ бит блоки $X = X_1 X_2 X_3 \ldots
X_B$ и $Y = Y_1 Y_2 Y_3 \ldots Y_B$ представляют собой блоки открытого и
закрытого текста соответственно. Линейный криптоанализ ищет линейное выражение
для некоторой комбинации входных и выходных бит
\begin{equation}\label{eq:linear_eq}\xorsum^{B}_{i=1} X_i \cdot \alpha_i =
\xorsum^{B}_{j=1} Y_j \cdot \beta_j,\end{equation}
где $\alpha, \beta \in \{0,1\}$. Лучшая линейная аппроксимация "--- это
выражение с наивысшей вероятностью появления, а наилучшая аффинная аппроксимация
"--- это выражение с наименьшей вероятностью появления. Пусть $P = P(X = Y)$
"--- вероятность, связанная с приведенным выше выражением~\ref{eq:linear_eq}.
Если $P \approx \frac{1}{2}$, это указывает на то, что шифр имеет высокую
стойкость к линейным и аффинным аппроксимациям. Смещение вероятности, задаваемое
как $|P - \frac{1}{2}|$, является отклонением от ожидаемой вероятности для
случайного процесса.

Любое линейное выражение, которое связывает биты открытого, закрытого текста и
ключа, должно учитывать структуру шифра и его компонентов, включая используемые
в раундах узлы замен. Для нахождения линейной аппроксимации S-блока $n \times
m$, линейные отношения между входами и выходами S-блока должны быть посчитаны
для всех пар входов и выходов, что выражается через матрицу $2^n \times 2^m$,
называемую таблицей линейных аппроксимаций LAT (Linear Approximation Table).

Каждое значение в ячейках таблицы линейных аппроксимаций $LAT_{X',Y'}$ S-блока
может быть посчитано как
\begin{equation}LAT_{X',Y'}=2^{n-1}-d_H(X', Y'),\end{equation}
где $d_H$ "--- расстояние по Хэммингу между двумя последовательностями. Это
значение дает смещение знаковой вероятности $\frac{L_{X', Y'}}{2^n} = P(X' = Y')
-\frac{1}{2}$ Смещение вероятности равное нулю указывает на то, что нет
возможных линейных аппроксимаций, в то время как смещение, достигающее $\pm
\frac{1}{2}$, указывает на то, что S-блок можно легко аппроксимировать с помощью
линейной или аффинной функции. Таким образом, лучшая линейная аппроксимация к
S-блоку $n \times m$ "--- это линейное выражение вида~\ref{eq:linear_eq}, чьи
входные и выходные биты, $X'$ и $Y'$ соответственно, относятся к наибольшему
значению таблицы линейных аппроксимаций S-блока.

Линейный криптоанализ блочных симметричных шифров включает нахождение линейной
аппроксимации с большой знаковой вероятностью на каждом шаге шифрования, т.е. на
каждом раунде. Объединение вероятностей линейных выражений, наилучшим образом
аппроксимирующих различные шаги процесса шифрования, основывается на
предположении о независимости линейных аппроксимаций на каждом шаге шифрования.
Общая линейная аппроксимация шифра получается путем связывания множества
линейных выражений вместе. Применение Леммы Мацуи (Piling-Up
Lemma)~\cite{Matsui} дает вероятность для линейной аппроксимации всего шифра.
Чем выше рассчитанная вероятность, тем более вероятно, что с помощью
аппроксимации удастся успешно получить биты ключа при достаточном количестве
заданных пар открытого-закрытого текста. Чем выше величина смещения, достигаемая
отдельными линейными выражениями на каждом шаге шифрования, тем выше общая
вероятность линейной аппроксимации шифра. Смещение больших значений в таблице
линейных аппроксимаций S-блока приводит к более успешной атаке на весь шифр.

Значения таблицы линейных аппроксимаций S-блока $n \times m$ тесно связаны со
значениями матрицы преобразования Уолша-Адамара всех линейных комбинаций
компонентных булевых функций S-блока. Смещение (bias) выражается через значения
преобразования Уолша-Адамара следующим образом:
\begin{equation}bias = L_{X', Y'} = \frac{\hat{F}(w)}{2},\end{equation}
что напрямую связано с нелинейностью S-блока:
\begin{equation}NL(S_{n,m}) = 2^{n-1} - |L_{X', Y'}|_{max},\end{equation}
где $X' \neq 0$, $Y' \neq 0$ и $|L_{X',Y'}|_{max}$ представляет максимальное
абсолютное значение в таблице линейных аппроксимаций.

Таким образом, использование высоконелинейных S-блоков в системах шифрования
предпочтительно для того, чтобы шифр имел стойкость к атакам линейного
криптоанализа. В 1993 г. в~\cite{Matsui} показано, что DES также был взломан
при помощи линейного криптоанализа. Успешность атаки была обусловлена наличием
больших значений в таблицах линейных аппроксимаций S-блоков DES. Как упоминалось
ранее, стойкость относительно линейного криптоанализа требует низких значений в
таблице линейных аппроксимаций, которые получаются через использование высоко
нелинейных S-блоков. Мы заключаем, что высокая нелинейность и низкая
автокорреляция "--- важные свойства S-блоков, необходимые для обеспечения
стойкости БСШ относительно дифференциального и линейного криптоанализа.

\section{Эффективность блочных симметричных шифров относительно
дифференциального и линейного криптоанализа}

Задачу проектирования практического алгоритма БСШ следует рассматривать как
задачу минимизации ''затрат'' на реализацию криптопреобразования,
обеспечивающего необходимые показатели стойкости~\cite{Golovashich}. При этом
можно утверждать, что шифр имеет стойкость относительно какого-либо вида
криптоанализа, если для успешной реализации атаки на шифр потребуется большее
число вычислительных затрат, чем на реализацию атаки типа ''грубой силы''.
Тогда можно утверждать, что исследуемый шифр достиг свойств случайной
подстановки относительно данного вида атак.

Общим требованием к разрабатываемым вычислительным методам синтеза нелинейных
узлов замен с улучшенными свойствами является улучшение показателей нелинейности
и автокорреляции S-блоков, что значит максимизировать нелинейность и
минимизировать актокорреляцию.

Общепринятыми показателями оценки стойкости алгоритмов БСШ относительно
дифференциального и линейного криптоанализа являются оценки вероятностей
дифференциальной и линейной характеристик шифра, которые необходимо
минимизировать.

При этом эффективность БСШ относительно дифференциального и линейного
криптоанализа определяется числом раундов зашифрования (либо вычислительных
затрат на реализацию шифрования), необходимых для выхода шифра к
дифференциальным и линейным свойствам случайных подстановок.

Очевидно, что чем меньше требуется вычислительных затрат на реализацию
криптопреобразования для обеспечения требуемых показателей стойкости, тем выше
эффективность шифра.

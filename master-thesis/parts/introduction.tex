\chapter*{Введение}

Нелинейные узлы замен (S-блоки) являются ключевым элементом современных
симметричных криптографических алгоритмов, среди которых наиболее известными
являются стандарты блочного симметричного шифрования DES, ГОСТ-28147-89 и AES. 

S-блок nxm представляет собой нелинейную функцию отображения n входных бит в m
выходных. В случае, когда на выходе S-блока появляются все возможные m-битные
значения и каждое выходное значение равновероятно, S-блок называют регулярным,
что является обязательным условием для его использования в современных БСШ.
Кроме того, для обеспечения устойчивости БСШ к атакам дифференциального и
линейного криптоанализа используемые узлы замен должны удовлетворять требуемым
показателям нелинейности и автокорреляции.

Вычислительным методам синтеза криптографически стойких регулярных узлов замен
посвящено множество работ. Однако, как показал проведенный анализ, показатели
формируемых S-блоков далеки от оптимальных, с увеличением их размерности
некоторые методы становятся вычислительно нереализуемыми.  Перспективным
направлением исследований является использование математического аппарата
недвоичных криптографических функций, что, как показано в данной работе,
позволяет синтезировать регулярные нелинейные узлы замен c требуемыми
показателями нелинейности и автокорреляции.

В данной работе исследуются новые критерии вычислительного поиска
криптографически стойких S-блоков. На основе усовершенствованных весовых
коэффициентов ценовых функций поиска предлагается дальнейшее развитие метода
имитации отжига.

В разделе ''Охрана труда и безопасность в чрезвычайных ситуациях'' будут
исследованы вопросы условий труда, безопасности, производсвтенной санитарии и
пожарной безопасности в помещении научно-исселедовательской лаборатории. Также
будет произведён расчёт необходимой площади световых проёмов в помещении
научно-исследовательской лаборатории.

\chapter*{Выводы}

Полученные результаты экспериментальных исследований эффективности метода
вычислительного поиска с использованием ценовых функций стоимости показывают,
что направление исследования нелинейных узлов замен является актуальным.
Использование предложенных динамических весовых коэффициентов в предлагаемых
функциях стоимости позволяют существенно повысить эффективность метода имитации
отжига "--- получены лучшие известные на сегодняшний день результаты по
нелинейности и автокорреляции для S-блоков $8 \times 2$. Для S-блоков 6x4
удалось поднять значение показателя нелинейности.  Таким образом, разработанный
вычислительный метод позволяет формировать нелинейные узлы замен с улучшенными
свойствами и использовать их для совершенствования DES-подобных симметричных
криптоалгоритмов.

Перспективным направлением дальнейших исследований является развитие
математического аппарата криптографических недвоичных функций для задач синтеза
биективных S-блоков, экспериментальные исследования эффективности предлагаемого
подхода для узлов замен больших размерностей, обобщение введенных в данной
работе динамических весовых коэффициентов.

В разделе ''Охрана труда и безопасность в чрезвычайных ситуациях'' был проведён
анализ условий труда исследователя на своём рабочем месте. Исследованы
промышленная безопасность в помещении научно-исследовательской лаборатории,
производственная санитария, а также производственная санитария в данном
помещении. Так же была расчитана нормативная площадь световых проёмов для
помещения НИЛ.
